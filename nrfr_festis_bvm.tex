% !TEX TS-program = lualatex
% !TEX encoding = UTF-8

\documentclass[nocturnal_bvm_fr.tex]{subfiles}

\ifcsname preamble@file\endcsname
  \setcounter{page}{\getpagerefnumber{M-nrfr_festif_bvm}}
\fi

\begin{document}
\feast{CBMV}{Propre de certaines fêtes de la Bienheureuse Vierge Marie}
	{Propre}{Propre}{2}{}{}{}{}{}{}
\thispagestyle{empty}

\rubric{Aux fêtes de l'Apparition de Notre-Dame à Lourdes, de l'Immaculée Conception, du Saint Rosaire et des Sept Douleurs, rien n'est du Commun. 
Ces fêtes font l'objet de livrets à part. Aux fêtes ci-après, tout est du Commun, sauf ce qui est donné au Propre.}

\feast{0202}{Purification de la Bienheureuse Vierge Marie}
	{Propre}{2 février}{2}{2 février}
	{}{}{Mariæ!Purificatio}
	{}
	{}
\gscore{0202I}{I}{}{Ecce venit ad templum}{Voici que le Seigneur Dominateur vient dans son saint temple : réjouis-toi et sois dans l’allégresse, Sion, en accourant au-devant de ton Dieu.}
\nocturn{1}
\resp{1}{Adórna thálamum tuum, Sion, et súscipe Regem Christum:
\psstar{} Quem virgo concépit, virgo péperit, virgo post partum, quem génuit, adorávit.
\vvrub{} Accípiens Símeon púerum in mánibus, grátias agens benedíxit Dóminum.}{Décore ta chambre nuptiale, ô Sion, et reçois le Christ-Roi :
\psstar{} Celui, que vierge elle a conçu, que vierge elle a enfanté, demeurant vierge après l’enfantement, celui qu’elle a mis au monde, elle l’a adoré.
\vvrub{} Recevant l’enfant dans ses bras, Siméon, avec actions de grâces, bénit le Seigneur.}
\resp{2}{Postquam impléti sunt dies purgatiónis Maríæ secúndum legem Móysi, tulérunt Jesum in Jerúsalem, ut sísterent eum Dómino,
\psstar{} Sicut scriptum est in lege Dómini: Quia omne masculínum adapériens vulvam, sanctum Dómino vocábitur.
\vvrub{} Obtulérunt pro eo Dómino par túrturum aut duos pullos columbárum.}{Après que furent accomplis les jours de la purification de Marie selon la loi de Moïse, ils portèrent Jésus à Jérusalem, afin de le présenter au Seigneur,
\psstar{} Comme il est écrit dans la loi du Seigneur : Tout mâle premier-né sera dit consacré au Seigneur.
\vvrub{} Ils offrirent pour lui au Seigneur une paire de tourterelles ou deux petits de colombes.}
\resp{3}{Obtulérunt pro eo Dómino par túrturum aut duos pullos columbárum.
\psstar{} Sicut scríptum est in lege Dómini.
\vvrub{} Postquam autem impléti sunt dies purgatiónis Maríæ secúndum legem Móysi, tulérunt illum in Jerúsalem, ut sísterent eum Dómino.}{Ils offrirent pour lui, au Seigneur, une paire de tourterelles ou deux petits de colombes.
\psstar{} Comme il est écrit dans la loi du Seigneur.
\vvrub{} Mais après que furent accomplis les jours de la purification de Marie, selon la loi de Moïse, ils le portèrent à Jérusalem, afin de le présenter au Seigneur.}
\nocturn{2}
\resp{4}{Símeon justus et timorátus exspectábat redemptiónem Israël,
\psstar{} Et Spíritus Sanctus erat in eo.
\vvrub{} Respónsum accépit Símeon a Spíritu Sancto, non visúrum se mortem, nisi vidéret Christum Dómini.}{Siméon, homme juste et craignant Dieu attendait la rédemption d’Israël,
\psstar{} Et l’Esprit-Saint était en lui.
\vvrub{} Siméon reçut cette réponse de l’Esprit-Saint, qu’il ne verrait point la mort avant d’avoir vu le Christ du Seigneur.}
\resp{5}{Respónsum accépit Símeon a Spíritu Sancto, non visúrum se mortem, nisi vidéret Christum Dómini:
\psstar{} Et benedíxit Deum, et dixit: Nunc dimíttis servum tuum in pace, quia vidérunt óculi mei salutáre tuum, Dómine.
\vvrub{} Cum indúcerent púerum Jesum paréntes ejus, ut fácerent secúndum consuetúdinem legis pro eo, ipse accépit eum in ulnas suas.}{Siméon reçut cette réponse de l’Esprit-Saint, qu’il ne verrait point la mort avant d’avoir vu le Christ du Seigneur :
\psstar{} Et il bénit Dieu et s’écria' ; Maintenant, Seigneur, laissez votre serviteur partir en paix, car mes yeux ont vu votre salut.
\vvrub{} Comme ses parents apportaient l’enfant Jésus, afin de faire ce que demandait la loi, lui-même le reçut dans ses bras.}
\resp{6}{Cum indúcerent púerum Jesum paréntes ejus in templum, ut fácerent secúndum consuetúdinem legis pro eo, accépit eum Símeon in ulnas suas, et benedíxit Deum, dicens:
\psstar{} Nunc dimíttis, Dómine, servum tuum in pace.
\vvrub{} Suscípiens Símeon Púerum in mánibus, exclamávit, dicens.}{Comme ses parents apportaient l’enfant Jésus au temple, afin de faire ce que demandait la loi, Siméon le reçut dans ses bras, et bénit Dieu, en disant :
\psstar{} Maintenant, Seigneur, laissez votre serviteur partir en paix.
\vvrub{} Siméon, prenant l’Enfant dans ses mains, s’écria : Maintenant.}
\nocturn{3}
\resp{7}{Suscípiens Jesum in ulnas suas Símeon, exclamávit, et dixit:
\psstar{} Tu es vere lumen ad illuminatiónem géntium, et glóriam plebis tuæ Israël.
\vvrub{} Cum indúcerent púerum Jesum paréntes ejus, et ipse accépit eum in ulnas suas, et benedíxit Deum, et dixit.}{Siméon prenant Jésus dans ses bras s’écria :
\psstar{} Vous êtes vraiment la lumière qui éclairera les nations, et la gloire d’Israël, notre peuple.
\vvrub{} Comme ses parents apportaient l’enfant Jésus, Siméon le reçut dans ses bras, et bénit Dieu, en disant :}
\resp{8}{Senex Púerum portábat, Puer autem senem regébat:
\psstar{} Quem virgo concépit, virgo péperit, virgo post partum, quem génuit, adorávit.
\vvrub{} Accípiens Símeon Púerum in mánibus, grátias agens benedíxit Dóminum.}{Le vieillard portait l’Enfant, mais l’Enfant guidait le vieillard :
\psstar{} Celui, que vierge elle a conçu, que vierge elle a enfanté, demeurant vierge après l’enfantement, celui qu’elle a mis au monde, elle l’a adoré.
\vvrub{} Siméon, prenant l’Enfant dans ses mains, bénit le Seigneur en rendant grâces.}

\feast{0325}{Annonciation à la Bienheureuse Vierge Marie}
	{Propre}{25 mars}{2}{25 mars}
	{}{}{Mariæ!Annuntiatio}
	{}
	{}
\gscore{0325I}{I}{}{Ave Maria gratia plena}{Je vous salue, Marie, pleine de grâce : le Seigneur est avec vous.}
\nocturn{1}
\resp{1}{Missus est Gábriel Angelus ad Maríam Vírginem desponsátam Joseph, núntians ei verbum; et expavéscit Virgo de lúmine: ne tímeas, María, invenísti grátiam apud Dóminum:
\psstar{} Ecce concípies et páries, et vocábitur Altíssimi Fílius.
\vvrub{} Dabit ei Dóminus Deus sedem David, patris ejus, et regnábit in domo Jacob in ætérnum.}{L’Ange Gabriel fut envoyé à Marie, vierge qu’avait épousée Joseph, lui portant un message, et la Vierge fut effrayée de la lumière. Ne craignez point, Marie ; vous avez trouvé grâce devant le Seigneur :
\psstar{} Voilà que vous concevrez et que vous enfanterez, et il sera appelé le Fils du Très-Haut.
\vvrub{} Le Seigneur Dieu lui donnera le trône de David, son père ; et il régnera éternellement sur la maison de Jacob.}
\resp{2}{Ave, María, grátia plena; Dóminus tecum:
\psstar{} Spíritus Sanctus supervéniet in te, et virtus Altíssimi obumbrábit tibi: quod enim ex te nascétur Sanctum, vocábitur Fílius Dei.
\vvrub{} Quómodo fiet istud, quóniam virum non cognósco? Et respóndens Angelus, dixit ei.}{Je vous salue, Marie, pleine de grâce ; le Seigneur est avec vous :
\psstar{} L’Esprit-Saint surviendra en vous, et la vertu du Très-Haut vous couvrira de son ombre ; c’est pourquoi la chose sainte qui naîtra de vous sera appelée le Fils de Dieu.
\vvrub{} Comment cela se fera-t-il ? Car je ne connais point d’homme. Et l’Ange répondant, lui dit.}
\resp{3}{Súscipe verbum, Virgo María, quod tibi a Dómino per Angelum transmíssum est: concípies et páries Deum páriter et hóminem,
\psstar{} Ut benedícta dicáris inter omnes mulíeres.
\vvrub{} Páries quidem fílium, et virginitátis non patiéris detriméntum: efficiéris grávida, et eris mater semper intácta.}{Recevez, Vierge Marie, la parole du Seigneur, qui vous est transmise par un Ange : vous concevrez et enfanterez Dieu et homme tout ensemble,
\psstar{} Et ainsi vous serez appelée bénie entre toutes les femmes.
\vvrub{} Vous enfanterez vraiment un fils, et votre virginité n’en souffrira point de détriment : vous concevrez, et vous serez mère toujours sans tache.}
\nocturn{2}
\resp{4}{Ecce virgo concípiet, et páriet fílium, dicit Dóminus:
\psstar{} Et vocábitur nomen ejus Admirábilis, Deus, Fortis.
\vvrub{} Super sólium David, et super regnum ejus sedébit in ætérnum.}{Voici, dit le Seigneur, que la Vierge concevra et enfantera un fils :
\psstar{} Et son nom sera appelé Admirable, Dieu, Fort.
\vvrub{} Il s’assiéra sur le trône de David, et sur son royaume, régnera pour l’éternité.}
\resp{5}{Egrediétur virga de radíce Jesse, et flos de radíce ejus ascéndet:
\psstar{} Et erit justítia cíngulum lumbórum ejus, et fides cinctórium renum ejus.
\vvrub{} Et requiéscet super eum Spíritus Dómini: spíritus sapiéntiæ et intelléctus, spíritus consílii et fortitúdinis.}{Il sortira un rejeton de la racine de Jessé, et une fleur s’élèvera de sa racine :
\psstar{} Et la justice sera la ceinture de ses reins, et la fidélité le ceinturon de ses flancs.
\vvrub{} Et l’esprit du Seigneur reposera sur lui, l’esprit de sagesse et d’intelligence, l’esprit de conseil et de force.}
\resp{6}{Sancta et immaculáta virgínitas, quibus te láudibus éfferam, néscio:
\psstar{} Quia quem cæli cápere non póterant, tuo grémio contulísti.
\vvrub{} Benedícta tu in muliéribus, et benedíctus fructus ventris tui.}{Sainte et immaculée virginité, je ne sais par quelles louanges vous exalter :
\psstar{} Car vous avez porté dans votre sein celui que les cieux ne peuvent contenir.
\vvrub{} Vous êtes bénie entre les femmes, et le fruit de votre sein est béni.}
\nocturn{3}
\resp{7}{Congratulámini mihi, omnes qui dilígitis Dóminum: quia cum essem párvula, plácui Altíssimo,
\psstar{} Et de meis viscéribus génui Deum et hóminem.
\vvrub{} Beátam me dicent omnes generatiónes, quia ancíllam húmilem respéxit Deus.}{Vous tous qui aimez le Seigneur, réjouissez-vous avec moi, parce que, comme j’étais petite, j’ai plu au Très-Haut :
\psstar{} Et de mes entrailles j’ai enfanté le Dieu-Homme.
\vvrub{} Toutes les générations me diront bienheureuse, parce que Dieu a regardé son humble servante.}
\resp{8}{Gaude, María Virgo, cunctas hǽreses sola interemísti, quæ Gabriélis Archángeli dictis credidísti:
\psstar{} Dum Virgo Deum et hóminem genuísti, et post partum, Virgo, invioláta permansísti.
\vvrub{} Beáta es quæ credidísti: quia perfécta sunt ea, quæ dicta sunt tibi a Dómino.}{Réjouissez-vous, Vierge Marie, qui avez cru aux paroles de l’Archange Gabriel, vous seule avez détruit toutes les hérésies :
\psstar{} En enfantant, vierge, un Dieu-Homme, et en restant vierge sans tache après l’enfantement.
\vvrub{} Vous êtes bienheureuse, vous qui avez cru, car les choses qui vous ont été dites par le Seigneur se sont accomplies.}

\feast{0531b}{La Bienheureuse Vierge Marie, Reine}
	{Propre}{31 mai}{2}{31 mai}
	{}{}{Mariæ!Regina}
	{}
	{}
\gscore{0531I}{I}{}{Christum Regem qui suam}{Le Christ Roi, qui couronna sa Mère, venez, adorons-le, alléluia.}
\gscore{0531H}{H}{}{Rerum supremo}{%
\rubric{1.} Au sommet de la création, en Reine,
ô Vierge, vous vous dressez,
surabondamment enrichie
par la beauté de l’univers.\\\\
\rubric{2.} Sa première œuvre, la plus belle,
vous brillez avant le Verbe créateur,
prédestinée à engendrer
le Fils qui vous a créée.\\\\
\rubric{3.} Comme le Christ, sur l’arbre sublime,
est Roi dans sa pourpre sanglante,
ainsi, prenant part à la passion,
vous êtes la Mère des vivants.\\\\
\rubric{4.} Ornée de si grandes gloires,
regardez vers nous, qui vous acclamons,
et recevez la louange reconnaissante
que nous chantons en votre honneur.\\\\
\rubric{5.} Gloire à toi, Seigneur,
qui es né de la vierge,
ainsi qu’au Père et à l’Esprit Saint,
dans les siècles éternels.
Amen.}

\nocturn{1}
\versiculus{Salve, Regina misericórdiæ, allelúia.}{Ex qua natus est Christus, Rex noster, allelúia.}{Salut, Reine de miséricorde, alléluia.}{De qui est né le Christ, notre Roi, alléluia.}
\resp{1}{Beáta es, María, quæ credidísti Dómino: perfécta sunt in te quæ dicta sunt tibi.
\psstar{} Ecce exaltáta es super choros Angelórum ad cæléstia regna, allelúja.
\vvrub{} Ave Maria, grátia plena, Dominus tecum.}{Bienheureuse êtes-vous, Marie, qui avez cru le Seigneur : elles se sont accomplies en vous, les annonces qui vous ont été faites.
\psstar{} Voici que vous avez été élevée au-dessus des chœurs des Anges, aux royaumes célestes. Alléluia.
\vvrub{} Salut, Marie, pleine de grâce, le Seigneur est avec vous.}
\resp{2}{Regálem dignitátem Vírginis Maríæ recolámus.
\psstar{} Quia cum Christo regnat in ætérnum. Allelúja.
\vvrub{} Glóriam Regínæ nostræ celebrémus.}{Rappelons la dignité de la Vierge Marie.
\psstar{} Car, avec le Christ, elle règne à jamais. alleluia.
\vvrub{} Célébrons la gloire de notre Reine.}
\resp{3}{Elégit eam Deus, et præelégit eam;
\psstar{} Corónam glóriæ pósuit super caput ejus. Allelúja.
\vvrub{} Et in tabernáculo suo habitáre fecit eam.}{Dieu l’a choisie, et l’a prédestinée
\psstar{} Il a posé sur sa tête la couronne de gloire. alleluia.
\vvrub{} Et sous sa tente il la fit habiter.}
\nocturn{2}
\versiculus{Stabat juxta crucem Jesu Mater ejus, allelúia.}{In passióne sócia, totíus mundi Regina, allelúia.}{Près de la croix de Jésus, sa mère se tenait debout, alléluia.}{Associée à sa passion, Reine du monde entier, alléluia.}
\resp{4}{Súscipe verbum, Virgo María, quod tibi a Dómino transmíssum est:
\psstar{} Ecce concípies et páries Deum páriter et hóminem. Allelúja.
\vvrub{} Et Regína vocáberis super omnes gentes.}{Recevez la parole, Vierge Marie, qui vous a été apportée de la part du Seigneur :
\psstar{} Voici que vous concevrez et que vous enfanterez un Dieu qui est également homme. alleluia.
\vvrub{} Et vous recevrez le titre de Reine sur toutes les nations.}
\resp{5}{Ecce pósitus est hic in ruínam et resurrectiónem multórum,
\psstar{} Et tuam ipsíus ánimam pertransíbit gládius. Allelúja.
\vvrub{} Ave, Christi Mater, passiónis sócia, totíus mundi Regína.}{Celui-ci est placé pour la chute et le relèvement d’un grand nombre.
\psstar{} Et vous-même, un glaive vous transpercera l’âme. alleluia.
\vvrub{} Salut, Mère du Christ, associée à la passion, Reine du monde entier.}
\resp{6}{Signum magnum appáruit in cælo:
\psstar{} Múlier amícta sole, et luna sub pédibus ejus, et in cápite ejus coróna stellárum duódecim. Allelúja.
\vvrub{} Cujus Fílius regnat in ætérnum.}{Un signe grandiose apparut au ciel :
\psstar{} Une Femme que le soleil enveloppe, la lune est sous des pieds, et douze étoiles couronnent sa tête. alleluia.
\vvrub{} Son Fils règne à jamais.}
\nocturn{3}
\versiculus{Beátam me dicent omnes generatiónes, allelúia.}{Quia fecit mihi magna qui potens est, allelúia.}{Tous les âges me diront bienheureuse, alléluia.}{Car le Puissant fit pour moi des merveilles, alléluia.}
\resp{7}{Ecce Virgo concípiet et páriet Fílium,
\psstar{} Et vocábitur nomen ejus Admirábilis, Deus, Fortis. Allelúja.
\vvrub{} Super sólium David et super regnum ejus sedébit in ætérnum.}{Voici que la Vierge concevra et enfantera un Fils.
\psstar{} Et il sera appelé Admirable, Dieu, Fort. alleluia.
\vvrub{} Sur le trône de David et sur son royaume il siégera éternellement.}
\resp{8}{Exaltáta es, sancta Dei Génitrix,
\psstar{} Super choros Angelórum ad cæléstia regna. Allelúja.
\vvrub{} Intercéde pro nobis ad Dóminum Jesum Christum.}{Vous avez été élevée, sainte Mère de Dieu,
\psstar{} Au-dessus des chœurs des Anges, aux célestes royaumes. Alléluia.
\vvrub{} Intercédez pour nous auprès du Seigneur Jésus Christ.}

\feast{0702}{Visitation de la Bienheureuse Vierge Marie}
	{Propre}{2 juillet}{2}{2 juillet}
	{}{}{Mariæ!Visitatio}
	{}
	{}
\gscore{0702I}{I}{}{Visitationem}{Célébrons la Visitation de la Vierge Marie, adorons son Fils, le Christ, notre Seigneur.}
\nocturn{1}
\resp{1}{Surge, própera, amíca mea, formósa mea, et veni: jam enim hiems tránsiit, imber ábiit et recéssit:
\psstar{} Vox túrturis audíta est in terra nostra.
\vvrub{} Intrávit María in domum Zacharíæ et salutávit Elísabeth.}{Lève-toi, hâte-toi, mon amie, ma toute belle, et viens ; car déjà l’hiver est passé, la pluie est partie, elle s’est retirée :
\psstar{} La voix de la tourterelle a été entendue dans notre terre.
\vvrub{} Marie entra dans la maison de Zacharie, et elle salua Élisabeth.}
\resp{2}{Quæ est ista quæ procéssit sicut sol, et formósa tamquam Jerúsalem?
\psstar{} Vidérunt eam fíliæ Sion, et beátam dixérunt, et regínæ laudavérunt eam.
\vvrub{} Et sicut dies verni circúmdabant eam flores rosárum et lília convállium.}{Quelle est celle-ci qui s’avance comme le soleil, et belle comme Jérusalem ?
\psstar{} Les filles de Sion l’ont vue et l’ont dite bienheureuse, et les reines l’ont louée.
\vvrub{} Et comme un jour de printemps, les rosés l’entouraient, ainsi que les lys des vallées.}
\resp{3}{Repléta est Spíritu Sancto Elísabeth et exclamávit: Benedícta tu inter mulíeres, et benedíctus fructus ventris tui:
\psstar{} Et unde hoc mihi, ut véniat mater Dómini mei ad me?
\vvrub{} Ecce enim, ut facta est vox salutatiónis tuæ in áuribus meis, exsultávit in gáudio infans in útero meo.}{Car, dès que la voix de votre salutation est venue à mes oreilles, l’enfant a tressailli de joie dans mon sein :
\psstar{} Et d’où m’arrive-t-il que la Mère de mon Seigneur vienne vers moi ?
\vvrub{} Car, dès que la voix de votre salutation est venue à mes oreilles, l’enfant a tressailli de joie dans mon sein.}
\nocturn{2}
\resp{4}{Ecce iste venit sáliens in móntibus, transíliens colles:
\psstar{} Símilis est diléctus meus cápreæ hinnulóque cervórum.
\vvrub{} Exsultávit ut gigas ad curréndam viam, a summo cælo egréssio ejus.}{Le voici qui vient, sautant sur les montagnes, franchissant les collines :
\psstar{} Mon bien-aimé est semblable au chevreuil et au faon des biches.
\vvrub{} Il s’est élancé comme un géant pour parcourir sa carrière ; à l’extrémité du ciel est sa sortie.}
\resp{5}{Congratulámini mihi, omnes qui dilígitis Dóminum: quia cum essem párvula, plácui Altíssimo,
\psstar{} Et de meis viscéribus génui Deum et hóminem.
\vvrub{} Beátam me dicent omnes generatiónes, quia ancíllam húmilem respéxit Deus.}{Vous tous qui aimez le Seigneur, réjouissez-vous avec moi, parce que comme j’étais petite, j’ai plu au Très-Haut :
\psstar{} Et de mes entrailles j’ai enfanté le Dieu-Homme.
\vvrub{} Toutes les générations me diront bienheureuse, parce que Dieu a regardé son humble servante.}
\resp{6}{Beáta quæ credidísti, quóniam perficiéntur in te quæ dicta sunt tibi a Dómino. Et ait María:
\psstar{} Magníficat ánima mea Dóminum.
\vvrub{} Veníte, et audíte, et narrábo quanta fecit Deus ánimæ meæ.}{Vous êtes bienheureuse, vous qui avez cru, car ce qui a été dit par le Seigneur s’accomplira en vous. Et Marie dit :
\psstar{} Mon âme glorifie le Seigneur.
\vvrub{} Venez et écoutez, et je raconterai quelles grandes choses Dieu a faites pour mon âme.}
\nocturn{3}
\resp{7}{Beátam me dicent omnes generatiónes,
\psstar{} Quia fecit mihi Dóminus magna qui potens est, et sanctum nomen ejus.
\vvrub{} Et misericórdia ejus a progénie in progénies timéntibus eum.}{Toutes les générations me diront bienheureuse,
\psstar{} Car le Seigneur, lui qui est puissant, m’a fait de grandes choses, et son nom est saint.
\vvrub{} Et sa miséricorde se répand d’âge en âge sur ceux qui le craignent.}
\resp{8}{Felix namque es, sacra Virgo María, et omni laude digníssima:
\psstar{} Quia ex te ortus est sol justítiæ, \psstar{} Christus, Deus noster.
\vvrub{} Ora pro pópulo, intérveni pro clero, intercéde pro devóto femíneo sexu: séntiant omnes tuum juvámen, quicúmque célebrant tuam sanctam Visitatiónem.}{Vous êtes heureuse, sainte Vierge Marie, et grandement digne de toute louange :
\psstar{} Car c’est de vous qu’est sorti le soleil de justice, \psstar{} le Christ notre Dieu.
\vvrub{} Priez pour le peuple, intervenez en faveur du clergé, intercédez pour les femmes consacrées par vœu ; qu’ils éprouvent tous votre assistance, ceux qui célèbrent votre sainte Visitation.}

\feast{0805}{Dédicace de Sainte-Marie-des-Neiges}
	{Propre}{5 août}{2}{5 août}
	{}{}{Mariæ!Dedicatio S. M. ad Nives}
	{}
	{}
\rubric{Tout du Commun}

\feast{0815}{Assomption de la Bienheureuse Vierge Marie}
	{Propre}{15 août}{2}{15 août}
	{}{}{Mariæ!Assumptio}
	{}
	{}
\gscore{0815I}{I}{}{Venite adoremus Regem}{Venez, adorons le Roi des rois, Qui a élevé au ciel, en ce jour, la Vierge sa Mère.}
\rubric{Hymne au Commun selon la coutume ancienne, ci-après selon la coutume récente.}
\gscore{0815H}{H}{}{Surge jam terris}{%
\rubric{1.} Levez-vous ! Déjà l’hiver cruel a quitté la terre,
dans les prés toute la beauté des fleurs sourit,
vous qui avez été la Mère nourricière de la Vie,
Levez-vous, Marie !\\\\
\rubric{2.} Lys resplendissant parmi les épines,
vous brisez seule l’auteur de la mort,
et le fruit refusé à nos pères coupables,
vous le cueillez à l’arbre de vie.\\\\
\rubric{3.} Arche faite d’un bois incorruptible,
vous conservez la manne d’où coule la force,
qui fait surgir des tombeaux
les ossements réanimés.\\\\
\rubric{4.} Docile servante de l’esprit qui préside,
la chair ne souffre pas d’être putréfiée ;
compagne à jamais de l’esprit,
elle s’élève au ciel.\\\\
\rubric{5.} Levez-vous ! Gagnez le ciel, appuyée sur le Bien-aimé,
recevez le diadème tressé d’étoile,
et accueillez le chant de vos fils
qui vous louent.\\\\
\rubric{6.} Louange éternelle à la très haute Trinité,
qui vous accorde, ô Vierge, la couronne,
et vous établit notre reine et notre mère,
dans sa providence. Ainsi soit-il.}

\nocturn{1}
\rubric{Psaumes du Commun}
\gscore{0815N1A1}{A}{1}{Exaltata est sancta}{Elle a été élevée au-dessus des chœurs des Anges, la sainte Mère de Dieu, dans le royaume céleste.}
\gscore{0815N1A2}{A}{2}{Paradisi portae}{Les portes du paradis nous ont été ouvertes grâce à vous, qui aujourd’hui triomphez glorieuse avec les Anges.}
\gscore{0815N1A3}{A}{3}{Benedicta tu in mulieribus}{Vous êtes bénie entre les femmes et le fruit de votre sein est béni.}
\versiculus{Exaltáta est sancta Dei Génitrix.}{Super choros Angelórum ad cæléstia regna.}{La sainte Mère de Dieu a été élevée.}{Au-dessus des chœurs des Anges, dans le royaume céleste.}
\resp{1}{Vidi speciósam sicut colúmbam, ascendéntem désuper rivos aquárum, cujus inæstimábilis odor erat nimis in vestiméntis ejus;
\psstar{} Et sicut dies verni circúmdabant eam flores rosárum et lília convállium.
\vvrub{} Quæ est ista quæ ascéndit per desértum sicut vírgula fumi ex aromátibus myrrhæ et thuris?}{Je l’ai vue belle comme une colombe qui s’élève au-dessus des rives des eaux ; un parfum inestimable se trouvait en abondance dans ses vêtements :
\psstar{} Et les fleurs des rosiers, et les lis des vallées l’entouraient comme un jour de printemps.
\vvrub{} Quelle est celle-ci, qui monte par le désert comme une colonne de fumée d’aromates, de myrrhe et d’encens ?}
\resp{2}{Sicut cedrus exaltáta sum in Líbano, et sicut cypréssus in monte Sion: quasi myrrha elécta,
\psstar{} Dedi suavitátem odóris.
\vvrub{} Et sicut cinnamómum et bálsamum aromatízans.}{Comme un cèdre, je me suis élevée sur le Liban, et comme un cyprès sur la montagne de Sion : comme la myrrhe de choix,
\psstar{} J’ai exhalé une odeur suave.
\vvrub{} Et comme le cinnamome et le baume aromatique.}
\resp{3}{Quæ est ista quæ procéssit sicut sol, et formósa tamquam Jerúsalem?
\psstar{} Vidérunt eam fíliæ Sion, et beátam dixérunt, et regínæ laudavérunt eam.
\vvrub{} Et sicut dies verni circúmdabant eam flores rosárum et lília convállium.}{Quelle est celle-ci qui s’avance comme le soleil, et belle comme Jérusalem ?
\psstar{} Les filles de Sion l’ont vue et l’ont dite bienheureuse, et les reines l’ont louée.
\vvrub{} Et comme un jour de printemps, les rosés l’entouraient, ainsi que les lys des vallées.}
\nocturn{2}
\rubric{Psalmodie du Commun}
\versiculus{Assúmpta est Maria in cælum: gaudent Angeli.}{Laudántes benedícunt Dóminum.}{Marie a été élevée au ciel : les Anges sont dans la joie.}{Ils louent et bénissent le Seigneur.}
\resp{4}{Ornátam monílibus fíliam Jerúsalem Dóminus concupívit:
\psstar{} Et vidéntes eam fíliæ Sion, beatíssimam prædicavérunt, dicéntes: Unguéntum effúsum nomen tuum.
\vvrub{} Astitit regína a dextris tuis in vestítu deauráto, circúmdata varietáte.}{Le Seigneur a aimé la fille de Jérusalem, parée de colliers :
\psstar{} Et les filles de Sion, la voyant, l’ont proclamée la plus heureuse, disant : \psstar{} C’est un parfum répandu que votre nom.
\vvrub{} La reine s’est tenue debout à votre droite, dans un vêtement d’or, couverte d’ornements variés.}
\resp{5}{Beátam me dicent omnes generatiónes,
\psstar{} Quia fecit mihi Dóminus magna qui potens est, et sanctum nomen ejus.
\vvrub{} Et misericórdia ejus a progénie in progénies timéntibus eum.}{Toutes les générations me diront bienheureuse,
\psstar{} Car le Seigneur, lui qui est puissant, m’a fait de grandes choses, et son nom est saint.
\vvrub{} Et sa miséricorde se répand d’âge en âge sur ceux qui le craignent.}
\resp{6}{Beáta es, Virgo María, quæ Dóminum portásti, Creatórem mundi:
\psstar{} Genuísti qui te fecit, et in ætérnum pérmanes Virgo.
\vvrub{} Ave, María, grátia plena; Dóminus tecum.}{Vous êtes bienheureuse, ô Vierge Marie, qui avez porté le Seigneur Créateur du monde :
\psstar{} Vous avez engendré Celui qui vous a faite, et vous demeurez Vierge à jamais.
\vvrub{} Je vous salue, Marie, pleine de grâce, le Seigneur est avec vous.}
\nocturn{3}
\rubric{Psalmodie du Commun}
\versiculus{María Virgo assúmpta est ad æthéreum thálamum.}{In quo Rex regum stelláto sedet sólio.}{La Vierge Marie a été élevée au séjour céleste.}{Où le Roi des rois est assis sur un trône étoilé.}
\resp{7}{Diffúsa est grátia in lábiis tuis:
\psstar{} Proptérea benedíxit te Deus in ætérnum.
\vvrub{} Myrrha, et gutta, et cásia a vestiméntis tuis, a dómibus ebúrneis, ex quibus delectavérunt te fíliæ regum in honóre tuo.}{La grâce est répandue sur vos lèvres :
\psstar{} C’est pourquoi le Seigneur vous a bénie pour l’éternité.
\vvrub{} La myrrhe, l’aloès et la cannelle s’exhalent de vos vêtements et de vos maisons d’ivoire, dont vous ont fait présent des filles de rois pour vous honorer.}
\resp{8}{Beáta es Virgo María, Dei Génitrix, quæ credidísti Dómino: perfécta sunt in te quæ dicta sunt tibi: ecce exaltáta es super choros Angelórum:
\psstar{} Intercéde pro nobis ad Dóminum, Deum nóstrum.
\vvrub{} Ave, María, grátia plena, Dóminus tecum.}{Vous êtes bienheureuse, Vierge Marie, Mère de Dieu, vous qui avez cru au Seigneur ; car ce qui vous a été dit, s’accomplira en vous : voici que vous êtes exaltée au-dessus des chœurs des Anges :
\psstar{} Intercédez pour nous auprès du Seigneur notre Dieu.
\vvrub{} Je vous salue, Marie, pleine de grâce, le Seigneur est avec vous.}

\feast{0822}{Le Cœur Immaculé de Marie}
	{Propre}{22 août}{2}{22 août}
	{}{}{Mariæ!Cor Immaculatum}
	{}
	{}
\rubric{Tout du Commun}

\feast{0908}{Nativité de la Bienheureuse Vierge Marie}
	{Propre}{8 septembre}{2}{8 septembre}
	{}{}{Mariæ!Nativitas}
	{}
	{}
\gscore{0908I}{I}{}{Nativitatem}{Célébrons la Nativité de la Vierge Marie : Adorons son Fils, le Christ Seigneur.}
\nocturn{1}
\resp{1}{Hódie nata est beáta Virgo María ex progénie David;
\psstar{} Per quam salus mundi credéntibus appáruit, cujus vita gloriósa lucem dedit sǽculo.
\vvrub{} Nativitátem beátæ Maríæ Vírginis cum gáudio celebrémus.}{Aujourd’hui est née la bienheureuse Vierge Marie de la descendance de David :
\psstar{} Par qui le salut du monde est apparu aux croyants et dont la vie glorieuse a donné lumière à ce siècle.
\vvrub{} Célébrons joyeusement la Nativité de la bienheureuse Vierge Marie.}
\resp{2}{Beatíssimæ Vírginis Maríæ Nativitátem devotíssime celebrémus,
\psstar{} Ut ipsa pro nobis intercédat ad Dóminum Jesum Christum.
\vvrub{} Cum jucunditáte Nativitátem beátæ Maríæ Vírginis devotíssime celebrémus.}{De la bienheureuse Vierge Marie, célébrons très dévotement la Nativité,
\psstar{} Pour qu’elle même intercède pour nous auprès du Seigneur Jésus-Christ.
\vvrub{} Avec bonheur célébrons très dévotement la Nativité de la bienheureuse Vierge Marie.}
\resp{3}{Gloriósæ Vírginis Maríæ ortum digníssimum recolámus,
\psstar{} Cujus Dóminus humilitátem respéxit, quæ, Angelo nuntiánte, concépit Salvatórem mundi.
\vvrub{} Beatíssimæ Vírginis Maríæ Nativitátem devotíssime celebrémus.}{De la glorieuse Vierge Marie, célébrons la très noble naissance
\psstar{} De celle dont le Seigneur a regardé la petitesse et qui, à l’annonce de l’Ange, a conçu le Sauveur du monde.
\vvrub{} De la bienheureuse Vierge Marie, célébrons très dévotement la Nativité.}
\nocturn{2}
\resp{4}{Natívitas gloriósæ Vírginis Maríæ ex semine Abrahæ, ortæ de tribu Juda, clara ex stirpe David;
\psstar{} Cujus vita ínclita cunctas illústrat ecclésias.
\vvrub{} Hódie nata est beáta Virgo María ex progénie David.}{C’est la Nativité de la glorieuse Vierge Marie, née de la race d’Abraham, de la tribu de Juda, de l’illustre famille de David :
\psstar{} De celle dont la glorieuse vie illustre toutes les Églises
\vvrub{} Aujourd’hui est née la bienheureuse Vierge Marie, de la race de David.}
\resp{5}{Cum jucunditáte Nativitátem beátæ Maríæ celebrémus,
\psstar{} Ut ipsa pro nobis intercédat ad Dóminum Jesum Christum.
\vvrub{} Corde et ánimo Christo canámus glóriam in hac sacra solemnitáte præcélsæ Genetrícis Dei Maríæ.}{Avec bonheur, célébrons la Nativité de la bienheureuse Vierge Marie
\psstar{} Pour qu’elle-même intercède pour nous, auprès du Seigneur Jésus-Christ.
\vvrub{} De cœur et d’esprit chantons gloire au Christ en cette solennité sacrée de la sublime Marie, Mère de Dieu.}
\resp{6}{Natívitas tua, Dei Génitrix Virgo, gáudium annuntiávit univérso mundo;
\psstar{} Ex te enim ortus est sol justítiæ, Christus Deus noster: \psstar{} Qui, solvens maledictiónem, dedit benedictiónem; et confúndens mortem, donávit nobis vitam sempitérnam.
\vvrub{} Benedícta tu in muliéribus, et benedíctus fructus ventris tui.}{Votre Nativité, ô Vierge Mère de Dieu, a annoncé la joie à tout l’univers ;
\psstar{} C’est de vous en effet, qu’est né le Soleil de justice, le Christ notre Dieu, \psstar{} Qui payant la dette de malédiction, nous a donné la bénédiction, et confondant la mort, nous a gratifié de la vie éternelle.
\vvrub{} Bénie êtes-vous entre les femmes et béni est le fruit de votre sein.}
\nocturn{3}
\resp{7}{Beátam me dicent omnes generatiónes,
\psstar{} Quia fecit mihi Dóminus magna qui potens est, et sanctum nomen ejus.
\vvrub{} Et misericórdia ejus a progénie in progénies timéntibus eum.}{Bienheureuse me diront toutes les générations ;
\psstar{} Car il a fait pour moi, de grandes choses, le Seigneur qui est puissant et dont saint est le nom.
\vvrub{} Et sa miséricorde s’étend de génération en génération, sur ceux qui le craignent.}
\resp{8}{Felix namque es, sacra Virgo María, et omni laude digníssima:
\psstar{} Quia ex te ortus est sol justítiæ, \psstar{} Christus, Deus noster.
\vvrub{} Ora pro pópulo, intérveni pro clero, intercéde pro devóto femíneo sexu: séntiant omnes tuum juvámen, quicúmque célebrant tuam sanctam Nativitátem.}{Heureuse êtes-vous, en effet, ô sainte Vierge Marie, et très digne de toute louange.
\psstar{} Puisque de vous est né le soleil de justice, \psstar{} Le Christ, notre Dieu.
\vvrub{} Priez pour le peuple, intervenez pour le clergé, intercédez pour les religieuses ; que tous ceux-là sentent votre secours, qui célèbrent votre sainte Nativité.}

\feast{0912}{Le Saint Nom de Marie}
	{Propre}{12 septembre}{2}{12 septembre}
	{}{}{Mariæ!Nomen}
	{}
	{}
\rubric{Tout du Commun}
	
\feast{1011}{Maternité de la Bienheureuse Vierge Marie}
	{Propre}{11 octobre}{2}{11 octobre}
	{}{}{Mariæ!Maternitas}
	{}
	{}
\gscore{1011I}{I}{}{Maternitatem}{Célébrons la Maternité de la bienheureuse Vierge Marie. Adorons le Christ, son Fils, Notre-Seigneur.}
\gscore{1011H}{H}{}{Caelo Redemptor}{%
\rubric{1.} Le Rédempteur a préféré au ciel,
le sein de la bienheureuse vierge ;
où, future victime,
il a revêtu un corps mortel.\\\\
\rubric{2.} Cette Vierge nous a enfanté
le guide de notre salut,
qui nous a rachetés de son sang,
et a souffert tourments et crucifiement.\\\\
\rubric{3.} Qu’un joyeux espoir, de notre cœur,
chasse les craintes anxieuses :
car nos larmes, cette Vierge les présente,
avec nos prières, à son Fils.\\\\
\rubric{4.} La voix de la Mère est bien reçue,
du Fils qui consent à ses vœux.
Que chacun donc aime toujours cette Mère,
et qu’aux heures difficiles, il l’invoque !\\\\
\rubric{5.} Gloire soit à vous, ô Trinité,
par qui le sein virginal de la Mère
a été enrichi d’un germe vivifiant.
Louange soit à vous, dans tous les siècles. Ainsi soit-il.
}

\nocturn{1}
\resp{1}{Felix es, sacra Virgo María et omni laude digníssima;
\psstar{} Ex qua ortus est sol justítiæ, Christus Deus noster, per quem salváti et redémpti sumus.
\vvrub{} Maternitátem beátæ Maríæ Vírginis cum gáudio celebrémus.}{Vous êtes heureuse, sainte Vierge Marie, et digne de toute louange,
\psstar{} Car par vous s’est levé le soleil de justice, le Christ Notre-Seigneur, par qui nous sommes sauvés et rachetés.
\vvrub{} Célébrons dans la joie la maternité de la sainte Vierge Marie.}
\resp{2}{Sine tactu pudóris invénta es Mater Salvatóris:
\psstar{} Qui cælum terrámque regit, in tua se clausit víscera factus homo.
\vvrub{} Benedícta tu in muliéribus, et benedíctus fructus ventris tui.}{Sans atteinte à votre virginité, vous êtes devenue mère du Sauveur,
\psstar{} Celui qui domine sur le ciel et la terre s’est enfermé en votre sein pour devenir homme
\vvrub{} Vous êtes bénie entre les femmes, et le fruit de vos entrailles est béni.}
\resp{3}{Multæ fíliæ congregavérunt divítias, tu supergréssa es univérsas:
\psstar{} Speciósa facta es et suávis in delíciis tuis, sancta Dei Génitrix.
\vvrub{} Séntiant omnes tuum juvámen, quicúmque célebrant tuam sanctam Maternitátem.}{Bien des filles se sont montrées vaillantes, mais vous, vous les surpasses toutes.
\psstar{} Vous êtes belle et douce en vos charmes, sainte Mère de Dieu.
\vvrub{} Qu’ils éprouvent votre secours, tous ceux qui célèbrent votre sainte maternité.}
\nocturn{2}
\resp{4}{Gloriósæ Vírginis Maríæ Maternitátem digníssimam recolámus:
\psstar{} Cujus Dóminus humilitátem respéxit, quæ Angelo nuntiánte concépit Salvatórem mundi.
\vvrub{} Christo canámus glóriam in hac sacra solemnitáte mirábilis Genetrícis Dei.}{Célébrons la très sainte maternité de la glorieuse Vierge Marie.
\psstar{} Le Seigneur a jeté les yeux sur sa petitesse, et à la parole de l’ange elle a conçu le Sauveur du monde.
\vvrub{} Chantons gloire au Christ en cette sainte solennité de l’admirable Mère de Dieu.}
\resp{5}{Benedícta fília tu a Dómino, quia per te fructum vitæ communicávimus:
\psstar{} Sola sine exémplo placuísti Dómino nostro Jesu Christo.
\vvrub{} Nostras deprecatiónes ne despícias in necessitátibus nostris, sed a perículis cunctis líbera nos, sancta Dei Génitrix.}{Vous êtes la fille bénie par le Seigneur, car par vous nous avons reçu le fruit de vie.
\psstar{} Seule, plus que toute autre vous avez été agréable à Jésus Christ notre Seigneur.
\vvrub{} Ne rejettez pas nos prières dans nos détresses, mais délivrez-nous de tous périls, sainte Mère de Dieu.}
\resp{6}{Benedícta tu inter mulíeres, et benedíctus fructus ventris tui:
\psstar{} Unde hoc mihi, ut véniat Mater Dómini mei ad me?
\vvrub{} Respéxit humilitátem ancíllæ suæ, et fecit mihi magna, qui potens est.}{Vous êtes bénie entre les femmes, et le fruit de vos entrailles est béni.
\psstar{} Comment m’est-il donné que la mère de mon Seigneur vienne à moi ?
\vvrub{} Il s’est penché sur son humble servante, et le Puissant a fait pour moi des merveilles.}
\nocturn{3}
\resp{7}{Beáta es Virgo María, Dei Génitrix, quæ credidísti Dómino: perfécta sunt in te, quæ dicta sunt tibi:
\psstar{} Proptérea benedíxit te Deus in ætérnum.
\vvrub{} Diffúsa est grátia in lábiis tuis: intercéde pro nobis ad Dóminum Deum nostrum.}{Vous êtes heureuse, Vierge Marie, Mère de Dieu, qui avez cru au Seigneur. Elles se sont accomplies, les choses qui vous ont été annoncées.
\psstar{} Aussi vous êtes bénie de Dieu à jamais.
\vvrub{} La grâce est répandue sur vos lèvres ; intercédez pour nous auprès du Seigneur, notre Dieu.}
\resp{8}{Congratulámini mihi, omnes qui dilígitis Dóminum: quia cum essem párvula, plácui Altíssimo:
\psstar{} Et de meis viscéribus génui Deum et hóminem.
\vvrub{} Beátam me dicent omnes generatiónes, quia ancíllam húmilem respéxit Deus.}{Félicitez-moi, vous tous qui aimez Dieu.
\psstar{} Car, toute petite, j’ai plu au Très-Haut, et de mon sein, j’ai enfanté Dieu fait homme.
\vvrub{} Tous les âges me diront bienheureuse, car Dieu s’est penché sur son humble servante.}

\feast{1121}{Présentation de la Bienheureuse Vierge Marie}
	{Propre}{21 novembre}{2}{21 novembre}
	{}{}{Mariæ!Præsentatio}
	{}
	{}
\rubric{Tout du Commun}

\end{document}