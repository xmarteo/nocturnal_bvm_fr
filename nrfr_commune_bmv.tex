% !TEX TS-program = lualatex
% !TEX encoding = UTF-8

\documentclass[nocturnal_bvm_fr.tex]{subfiles}

\ifcsname preamble@file\endcsname
  \setcounter{page}{\getpagerefnumber{M-nrfr_commune_bmv}}
\fi

\begin{document}
\feast{CBMV}{Commun de la Bienheureuse Vierge Marie}
	{Commun}{Commun}{2}{}{}{}{}{}{}
\thispagestyle{empty}


\intermediatetitle{Avant l'Office}

\begin{paracol}{2}
\lettrine{A}{peri}, Dómine, os meum ad benedicéndum nomen sanctum tuum: munda quoque cor meum ab ómnibus vanis, pervérsis et aliénis 
cogitatiónibus; intelléctum illúmina, afféctum inflámma, ut digne, atténte ac devóte hoc officium recitáre váleam, et exaudíri mérear
ante conspéctum divinae Majestátis tuae. Per Christum Dóminum nostrum. Amen.

\switchcolumn

\lettrine{O}{uvre} ma bouche, Seigneur, afin qu’elle bénisse ton saint nom, purifie aussi mon cœur de toute pensée
vaine, mauvaise, étrangère. Éclaire mon intelligence, enflamme mon amour,
afin que je puisse réciter cet office avec respect, attention et dévotion, et mériter d’être exaucé en
présence de ta divine majesté. Par le Christ, notre Seigneur. Amen.

\switchcolumn*

\lettrine{D}{ómine}, in unióne illíus divínæ intentiónis, qua ipse in terris laudes Deo persolvísti, has tibi horas \rubric{(vel} hanc tibi horam\rubric{)} persólvo.

\switchcolumn

\lettrine{S}{eigneur}, en union avec ces divines intentions que tu avais toi-même sur terre lorsque
tu louais Dieu, je t’offre cette \rubric{(}ces\rubric{)} heure\rubric{(}s\rubric{)}.

\switchcolumn*

\lettrine{P}{ater noster}, qui es in cælis, sanctificétur nomen tuum. Advéniat regnum tuum. Fiat volúntas tua, sicut in cælo et in terra.
Panem nostrum quotidiánum da nobis hódie. Et dimítte nobis débita nostra, sicut et nos dimíttimus debitóribus nostris. Et ne nos
indúcas in tentatiónem: sed líbera nos a malo. Amen.

\switchcolumn

\lettrine{N}{otre Père}, qui es aux cieux, que ton nom soit sanctifié, que ton règne vienne,
que ta volonté soit faite sur la terre comme au ciel.
Donne-nous aujourd’hui notre pain de ce jour. Pardonne-nous nos offenses, comme nous pardonnons aussi à ceux qui nous ont offensés.
Et ne nous laisse pas entrer en tentation mais délivre-nous du Mal. Amen.

\switchcolumn*

\lettrine{A}{ve Maria}, grátia plena, Dóminus tecum: benedícta tu in muliéribus, 
et benedíctus fructus ventris tui Jesus. Sancta María, Mater Dei, ora pro nobis 
peccatóribus, nunc et in hora mortis nostræ. Amen.

\switchcolumn

\lettrine{J}{e vous salue Marie}, pleine de grâce, le Seigneur est avec vous.
Vous êtes bénie entre toutes les femmes et Jésus, le fruit de vos entrailles, est béni.
Sainte Marie, Mère de Dieu, priez pour nous pauvres pécheurs, maintenant et à l’heure de notre mort.
Amen.

\switchcolumn*

\lettrine{C}{redo in Deum}, Patrem omnipoténtem, Creatórem cæli et terræ. Et in Jesum Christum, Fílium ejus únicum, Dóminum nostrum,
qui concéptus est de spíritu Sancto, natus ex María Virgine, passus sub Póntio Piláto, crucifixus, mórtuus et sepúltus: descéndit
ad ínferos: tértia die resurréxit a mórtuis; ascéndit ad cælos, sedet ad déxteram Patris omnipoténtis: inde ventúrus est judicáre
vivos et mórtuos. Credo in Spíritum sanctum, sanctam Ecclésiam cathólicam, Sanctórum communiónem, remissiónem peccatórum,
carnis resurrectiónem, vitam ætérnam. Amen.

\switchcolumn

\lettrine{J}{e crois en Dieu}, le Père tout-puissant, créateur du ciel et de la terre.
Et en Jésus Christ, son Fils unique, notre Seigneur;
qui a été conçu du Saint Esprit, est né de la Vierge Marie,
a souffert sous Ponce Pilate, a été crucifié,
est mort et a été enseveli, est descendu aux enfers;
le troisième jour est ressuscité des morts,
est monté aux cieux, est assis à la droite de Dieu le Père tout-puissant,
d’où il viendra juger les vivants et les morts.
Je crois en l’Esprit Saint, à la sainte Église catholique, à la communion des saints,
à la rémission des péchés, à la résurrection de la chair, à la vie éternelle. Amen.

\end{paracol}

\intermediatetitle{Ouverture de l'Office}

\gscore{ORIa}{T}{}{Domine labia mea!Tonus simplex}{Seigneur, ouvre mes lèvres, et ma bouche annoncera ta louange.\\\\
Dieu, viens à mon aide, Seigneur, hâte-toi de me secourir.\\\\
Gloire au Père, et au Fils, et au Saint-Esprit\\
comme il était au commencement, maintenant et toujours, pour les siècles des siècles. Amen.\\
Alléluia.\\
\rubric{Septuagésime et Carême:} Louange à toi, Seigneur, Roi d'éternelle gloire.}

\smalltitle{Invitatoire}
\gscore{CBMVI}{I}{}{Sancta Maria}{Sainte Marie, Mère de Dieu et Vierge, Intercédez pour nous.}
\label{M-ORIP2d}\label{M-ORIP3c}\label{M-ORIP3e}\label{M-ORIP4d}\label{M-ORIP4e}\label{M-ORIP4g}\label{M-ORIP5g}\label{M-ORIP6a}\label{M-ORIP6f}\label{M-ORIP7a}\label{M-ORIP7g}
\psaume{94}{VLrepet}

\smalltitle{Hymne}
\gscore{CBMVH}{H}{}{Quem terra pontus aethera}{%
\rubric{1.} Celui que terre, mer, astres
vénèrent, adorent, annoncent,
celui qui régit ce triple monde,
Marie le porte caché dans son sein.\\\\
\rubric{2.} Celui que lune, soleil et toutes choses
servent en tout temps,
est porté par les entrailles d’une jeune vierge,
toute pénétrée de la grâce céleste.\\\\
\rubric{3.} La bienheureuse mère, par la grâce,
dans l’arche de son sein,
renferme l’Artisan suprême
qui tient le monde dans sa main.\\\\
\rubric{4.} Bienheureuse, à la parole d’un messager du ciel,
féconde par le Saint-Esprit,
et son sein donne au monde
le désiré des nations.\\\\
\rubric{5.} Gloire à toi, Seigneur,
qui es né de la vierge,
ainsi qu’au Père et à l’Esprit Saint,
dans les siècles éternels.
Amen.}

\nocturn{1}
\gscore{CBMVN1A1}{A}{1}{Benedicta tu in mulieribus}{Vous êtes bénie entre les femmes et le fruit de votre sein est béni.}
\tptresrubric
\psaume{8}{4e}
\gscore{CBMVN1A2}{A}{2}{Sicut myrrha}{Comme une myrrhe de choix vous avez exhalé un parfum suave, ô sainte Mère de Dieu.}
\psaume{18}{4e}
\gscore{MUVXN1A2}{A}{3}{Ante torum}{Devant le trône de cette Vierge, chantez-nous souvent de doux cantiques qui nous rappellent ses saintes actions.}
\psaume{23}{4e}
\versiculustpall{Spécie tua et pulchritúdine tua.\\}{Inténde, próspere procéde, et regna.\\}{Dans votre gloire et votre beauté.\\}{Avancez heureusement, avancez et régnez.}

\twoside{Pater noster... \rubric{en silence jusqu'à}}{Notre Père...}

\versiculus{Et ne nos indúcas in tentatiónem.}{Sed líbera nos a malo.}{Et ne nous laisse pas entrer en tentation.}{Mais délivre-nous du Mal.}

\smalltitle{Absolution}

\twoside{Exáudi, Dómine Jesu Christe, preces servórum tuórum,~\pscross{} et mise\textit{rére} \textbf{no}bis:~\psstar{} Qui cum Patre et Spíritu Sancto vivis et regnas in sǽcula sæculórum.\\ \rubric{\rrrub} Amen.}{Seigneur Jésus-Christ, exauce les prières de tes serviteurs, et aie pitié de nous,
toi qui vis et règnes avec le Père et le Saint-Esprit, dans les siècles des siècles.\\ \rubric{\rrrub} Amen.}


\smalltitle{Bénédictions, Leçons et Répons}

\twoside{\rubric{Lector:} Jube, domne, benedícere.}{\rubric{Lecteur:} Veuillez, maître, bénir.}
\rubric{Si le célébrant n'est pas au moins diacre, le lecteur demande cette bénédiction à la croix et dit \normaltext{Dómine} et non \normaltext{domne}.}

\twoside{%
\rubric{\emph{Benedictio 1.}} Benedictió\textit{ne per}\textbf{pé}tua~\GreSpecial{*}
benedícat nos Pater ætérnus.
\hspace{\specialcharhsep}\rr Amen.%
}{%
\rubric{Bén. 1} Que le Père éternel nous bénisse d'une bénédiction perpétuelle.\\
\hspace{\specialcharhsep}\rr Amen.%
}

\rubric{Toutes les leçons sont propres et ne figurent pas dans ce carnet. Après chaque leçon:}
\twoside{%
\rubric{Lector:} Tu autem, Dómine, miserére nobis.\\
\rr Deo grátias.%
}{%
\rubric{Lecteur:} Et toi, Seigneur, aie pitié de nous.\\ \rr Nous rendons grâces à Dieu.}

\resp{1}{Sancta et immaculáta virgínitas, quibus te láudibus éfferam, néscio:
\psstar{} Quia quem cæli cápere non póterant, tuo grémio contulísti.
\hspace{\specialcharhsep}\vv Benedícta tu in muliéribus, et benedíctus fructus ventris tui.}{Sainte et immaculée virginité, je ne sais par quelles louanges vous exalter :
\psstar{} Car vous avez porté dans votre sein celui que les cieux ne peuvent contenir.
\hspace{\specialcharhsep}\vv Vous êtes bénie entre les femmes, et le fruit de votre sein est béni.}

\twoside{%
\rubric{\emph{Benedictio 2.}} Unigénitus \textit{Dei} \textbf{Fí}lius~\GreSpecial{*}
nos benedícere et adjuváre dignétur.
\hspace{\specialcharhsep}\rr Amen.%
}{%
\rubric{\emph{Bén. 2.}} Que le Fils unique de Dieu daigne nous bénir et nous secourir.
\hspace{\specialcharhsep}\rr Amen.
}

\resp{2}{Congratulámini mihi, omnes qui dilígitis Dóminum: quia cum essem párvula, plácui Altíssimo,
\psstar{} Et de meis viscéribus génui Deum et hóminem.
\hspace{\specialcharhsep}\vv Beátam me dicent omnes generatiónes, quia ancíllam húmilem respéxit Deus.}{Vous tous qui aimez le Seigneur, réjouissez-vous avec moi, parce que comme j’étais petite, j’ai plu au Très-Haut
\psstar{} Et de mes entrailles j’ai enfanté le Dieu-Homme.
\hspace{\specialcharhsep}\vv Toutes les générations me diront bienheureuse, parce que Dieu a regardé son humble servante.}

\twoside{%
\rubric{\emph{Benedictio 3.}} Spíritus \textit{Sancti} \textbf{grá}tia~\GreSpecial{*}
illúminet sensus et corda nostra.
\hspace{\specialcharhsep}\rr Amen.%
}{%
\rubric{\emph{Bén. 3.}} Que la grâce du Saint-Esprit illumine nos esprits et nos cœurs.
\hspace{\specialcharhsep}\rr Amen.%
}

\resp{3}{Beáta es, Virgo María, quæ Dóminum portásti, Creatórem mundi:
\psstar{} Genuísti qui te fecit, et in ætérnum pérmanes Virgo.
\hspace{\specialcharhsep}\vv Ave, María, grátia plena; Dóminus tecum.}{Vous êtes bienheureuse, ô Vierge Marie, qui avez porté le Seigneur Créateur du monde :
\psstar{} Vous avez engendré Celui qui vous a faite, et vous demeurez Vierge à jamais.
\hspace{\specialcharhsep}\vv Je vous salue, Marie, pleine de grâce, le Seigneur est avec vous.}

\rubric{Après le dernier répons de chaque nocturne, on ajoute le \normaltext{Glória Patri}.}

\nocturn{2}
\gscore{MUXXN2A1}{A}{4}{Specie tua}{Dans votre dignité et votre beauté, avancez, avancez avec succès et régnez.}
\tptresrubric
\psaume{44}{7}
\gscore{MUXXN2A2}{A}{5}{Adjuvabit eam}{Dieu la protège de son regard : Dieu est au milieu d’elle, elle ne sera pas ébranlée.}
\psaume{45}{7}
\gscore{CBMVN2A3}{A}{6}{Sicut laetantium}{Comme celui de tous ceux qui possèdent la vraie joie, notre refuge est en vous, sainte Mère de Dieu.}
\psaume{86}{7}
\versiculustpall{Adjuvábit eam Deus vultu suo.\\}{Deus in médio ejus, non commovébitur.\\}{Dieu la protège de son regard.\\}{Dieu est au milieu d’elle, elle ne sera pas ébranlée.}

\twoside{Pater noster... \rubric{en silence jusqu'à}}{Notre Père...}

\versiculus{Et ne nos indúcas in tentatiónem.}{Sed líbera nos a malo.}{Et ne nous laisse pas entrer en tentation.}{Mais délivre-nous du Mal.}

\pagebreak

\twoside{%
\rubric{\emph{Absolutio 2.}}
Ipsíus píetas et misericódi\textit{a nos} \textbf{ád}juvet,~\GreSpecial{*}
qui cum Patre et Spíritu Sancto vivit et regnat in sǽcula sæculórum.
\hspace{\specialcharhsep}\rr Amen.%
}{%
\rubric{\emph{Absolution 2.}}
Qu'il nous secoure par sa bonté et sa miséricorde, celui qui, avec le Père et le Saint-Esprit, vit et règne dans les siècles des siècles.
\hspace{\specialcharhsep}\rr Amen.%
}

\smalltitle{Bénédictions, Leçons et Répons}

\twoside{%
\rubric{\emph{Benedictio 4.}} Deus Pa\textit{ter om}\textbf{ní}potens~\GreSpecial{*}
sit nobis propítius et clemens.
\hspace{\specialcharhsep}\rr Amen.%
}{%
\rubric{\emph{Bén. 4.}}
Que Dieu le Père tout-puissant soit pour nous propice et plein de clémence.
\hspace{\specialcharhsep}\rr Amen.%
}

\resp{4}{Sicut cedrus exaltáta sum in Líbano, et sicut cypréssus in monte Sion: quasi myrrha elécta,
\psstar{} Dedi suavitátem odóris.
\hspace{\specialcharhsep}\vv Et sicut cinnamómum et bálsamum aromatízans.}{Comme un cèdre, je me suis élevée sur le Liban, et comme un cyprès sur la montagne de Sion : comme la myrrhe de choix,
\psstar{} J’ai exhalé une odeur suave.
\hspace{\specialcharhsep}\vv Et comme le cinnamome et le baume aromatique.}

\twoside{%
\rubric{\emph{Benedictio 5.}} Chris\textit{tus per}\textbf{pé}tuæ~\GreSpecial{*}
det nobis gaúdia vitæ.
\hspace{\specialcharhsep}\rr Amen.%
}{%
\rubric{\emph{Bén. 5.}}
Que le Christ nous donne les joies de l'éternelle vie.
\hspace{\specialcharhsep}\rr Amen.%
}

\resp{5}{Quæ est ista quæ procéssit sicut sol, et formósa tamquam Jerúsalem?
\psstar{} Vidérunt eam fíliæ Sion, et beátam dixérunt, et regínæ laudavérunt eam.
\hspace{\specialcharhsep}\vv Et sicut dies verni circúmdabant eam flores rosárum et lília convállium.}{Quelle est celle-ci qui s’avance comme le soleil, et belle comme Jérusalem ?
\psstar{} Les filles de Sion l’ont vue et l’ont dite bienheureuse, et les reines l’ont louée.
\hspace{\specialcharhsep}\vv Et comme un jour de printemps, les rosés l’entouraient, ainsi que les lys des vallées.}

\twoside{%
\rubric{\emph{Benedictio 6.}} Ignem su\textit{i a}\textbf{mó}ris~\GreSpecial{*}
accéndat Deus in córdibus nostris.
\hspace{\specialcharhsep}\rr Amen.%
}{%
\rubric{\emph{Bén. 6.}}
Que Dieu daigne allumer dans nos cœurs le feu de son amour.
\hspace{\specialcharhsep}\rr Amen.%
}

\resp{6}{Ornátam monílibus fíliam Jerúsalem Dóminus concupívit:
\psstar{} Et vidéntes eam fíliæ Sion, beatíssimam prædicavérunt, dicéntes: \psstar{} Unguéntum effúsum nomen tuum.
\hspace{\specialcharhsep}\vv Astitit regína a dextris tuis in vestítu deauráto, circúmdata varietáte.}{Le Seigneur a aimé la fille de Jérusalem, parée de colliers :
\psstar{} Et les filles de Sion, la voyant, l’ont proclamée la plus heureuse, disant : \psstar{} C’est un parfum répandu que votre nom.
\hspace{\specialcharhsep}\vv La reine s’est tenue debout à votre droite, dans un vêtement d’or, couverte d’ornements variés.}


\nocturn{3}
\gscore{CBMVN3A1}{A}{7}{Gaude Maria Virgo}{Réjouissez-vous, Vierge Marie : vous seule avez détruit toutes les hérésies dans le monde entier.}
\tptresrubric
\psaume{95}{4e}
\gscore{CBMVN3A2}{A}{8}{Dignare me laudare}{Rendez-moi digne de vous louer, ô Vierge sainte ; donnez-moi de la force contre vos ennemis.}
\psaume{96}{4e}
\rubric{Hors de l'Avent:}
\gscore{CBMVN3A3a}{A}{9}{Post partum Virgo}{Après l’enfantement, ô Vierge vous êtes demeurée dans votre intégrité première ; Mère de Dieu, intercédez pour nous.}
\psaume{97}{4e}
\vspace{\baselineskip}
\rubric{En Avent et à l'Annonciation:}
\gscore{CBMVN3A3b}{A}{9}{Angelus Domini nuntiavit}{L’Ange du Seigneur, annonça à Marie et elle conçut de l’Esprit-Saint.}
\psaume{97}{1}
\versiculustpall{Elégit eam Deus, et præelégit eam.\\}{In tabernáculo suo habitáre facit eam.\\}{Dieu l’a élue et choisie avec prédilection.}{Il l’a fait habiter dans son tabernacle.\\}

\twoside{Pater noster... \rubric{en silence jusqu'à}}{Notre Père...}

\versiculus{Et ne nos indúcas in tentatiónem.}{Sed líbera nos a malo.}{Et ne nous laisse pas entrer en tentation.}{Mais délivre-nous du Mal.}

\twoside{%
\rubric{\emph{Absolutio 3.}}
A vínculis peccató\textit{rum nos}\textbf{tró}rum~\GreSpecial{*}
absólvat nos omnípotens et miséricors Dóminus.
\hspace{\specialcharhsep}\rr Amen.%
}{%
\rubric{\emph{Absolution 3.}}
Que le Dieu tout-puissant et miséricordieux daigne nous délivrer des liens de nos péchés.
\hspace{\specialcharhsep}\rr Amen.%
}

\smalltitle{Bénédictions, Leçons et Répons}

\twoside{%
\rubric{\emph{Benedictio 7.}}
Evangé\textit{lica} \textbf{léc}tio~\GreSpecial{*}
sit nobis salus et protéctio.
\hspace{\specialcharhsep}\rr Amen.%
}{%
\rubric{\emph{Bén. 7.}}
Que la lecture du saint Évangile nous soit salut et protection.
\hspace{\specialcharhsep}\rr Amen.%
}

\resp{7}{Felix namque es, sacra Virgo María, et omni laude digníssima:
\psstar{} Quia ex te ortus est sol justítiæ, Christus, Deus noster.
\hspace{\specialcharhsep}\vv Ora pro pópulo, intérveni pro clero, intercéde pro devóto femíneo sexu: séntiant omnes tuum juvámen, quicúmque célebrant tuam sanctam festivitátem.}{Vous êtes heureuse, sainte Vierge Marie, et grandement digne de toute louange :
\psstar{} Car c’est de vous qu’est sorti le soleil de justice, le Christ notre Dieu.
\hspace{\specialcharhsep}\vv Priez pour le peuple, intervenez en faveur du clergé, intercédez pour les femmes consacrées par vœu ; qu’ils éprouvent tous votre assistance, ceux qui célèbrent votre sainte fête.}

\twoside{%
\rubric{\emph{Benedictio 8.}}
Cujus \textit{festum} \textbf{có}limus,~\GreSpecial{*}
ipsa Virgo vírginum intercédat pro nobis ad Dóminum.
\hspace{\specialcharhsep}\rr Amen.%
}{%
\rubric{\emph{Bén. 8.}}
Que celle dont nous célébrons la fête, la Vierge des vierges elle-même, intercède pour nous auprès du Seigneur.
\hspace{\specialcharhsep}\rr Amen.%
}

\resp{8}{Beátam me dicent omnes generatiónes,
\psstar{} Quia fecit mihi Dóminus magna qui potens est, et sanctum nomen ejus.
\hspace{\specialcharhsep}\vv Et misericórdia ejus a progénie in progénies timéntibus eum.}{Toutes les générations me diront bienheureuse,
\psstar{} Car le Seigneur, lui qui est puissant, m’a fait de grandes choses, et son nom est saint.
\hspace{\specialcharhsep}\vv Et sa miséricorde se répand d’âge en âge sur ceux qui le craignent.}

\twoside{%
\rubric{\emph{Benedictio 9.}}
Ad societátem cívium \textit{super}\textbf{nó}rum~\GreSpecial{*}
perdúcat nos Rex Angelórum.
\hspace{\specialcharhsep}\rr Amen.%
}{%
\rubric{\emph{Bén. 9.}}
Que le Roi des Anges nous fasse parvenir à la société des citoyens célestes.
\hspace{\specialcharhsep}\rr Amen.%
}

\rubric{Si on lit un évangile à la dernière leçon en vertu d'une commémoraison:}

\twoside{%
\rubric{\emph{Benedictio 9.}}
Per evangé\textit{lica} \textbf{dic}ta~\GreSpecial{*}
deleántur nostra delícta.
\hspace{\specialcharhsep}\rr Amen.%
}{%
\rubric{\emph{Bén. 9.}}
Par les paroles de l'Évangile, que nos péchés soient effacés.
\hspace{\specialcharhsep}\rr Amen.%
}

\intermediatetitle{Te Deum}

\newcommand{\tedeumtranslation}{Nous te louons ô Dieu : nous te reconnaissons pour le Seigneur.
Ô Père éternel, toute la terre te révère.
Tous les Anges les Cieux, et toutes les Puissances,
les Chérubins et les Séraphins te proclament sans cesse :\\\\\\\\
Saint, Saint, Saint le Seigneur, le Dieu des armées.
Les Cieux et la terre sont remplis de la majesté de ta gloire.\\\\\\\\
Le chœur glorieux des Apôtres,
le phalange vénérable des Prophètes,
l'armée des Martyrs éclatante de blancheur célèbre tes louanges;\\\\\\\\
La sainte Église confesse ton nom par toute la terre,
ô Père d'infinie majesté!
Et elle vénère ton Fils véritable et unique,
ainsi que le Saint-Esprit consolateur.\\\\\\\\
Tu es le Roi de gloire ô Christ!
Tu es du Père le Fils éternel.\\\\\\\\
\rubric{À \normaltext{Tu ad liberandum}, on s'incline.}\\
Pour délivrer l'homme, tu n'as pas eu horreur du sein d'une Vierge.
Tu as brisé l'aiguillon de la mort et ouvert aux fidèles le royaume des cieux.
Tu es assis à la droite de Dieu dans la gloire du Père.
Nous croyons que tu es le juge qui doit venir.\\\\\\\\
\rubric{À \normaltext{Te ergo}, on s'agenouille.}\\
Nous te supplions donc de secourir tes serviteurs que tu as rachetés par ton Sang précieux.
Fais qu'ils soient au nombre des saints, dans la gloire éternelle.\\\\\\\\
Sauve ton peuple, Seigneur  et bénis ton héritage.
Conduis tes serviteurs et élèves-les jusque dans l'éternité.
Chaque jour nous te bénissons.
Et nous louons ton nom dans les siècles; et dans les siècles des siècles.\\\\\\\\
Daigne Seigneur, en ce jour nous préserver de tout péché.
Aie pitié de nous Seigneur, aie pitié de nous.\\\\\\\\
Que ta miséricorde, Seigneur se répande sur nous, comme notre espérance est en toi.
J'ai éspéré en toi Seigneur ; que je ne sois pas confondu à jamais.}

\rubric{Ton simple}

\gscore{ORTDa}{H}{}{Te Deum laudamus!Ton simple}{%
\tedeumtranslation
}

\rubric{Ton solennel}

\gscore{ORTDb}{H}{}{Te Deum laudamus!Ton solennel}{
\tedeumtranslation
}


\intermediatetitle{Conclusion}

\rubric{Après le \normaltext{Te Deum}, si les Matines ne sont pas immédiatement suivies des Laudes, l'officiant dit:}

\versiculus{Dóminus vobíscum.}{Et cum spíritu tuo.}{Le Seigneur soit avec vous.}{Et avec votre esprit.}

\rubric{Ou bien, si l'officiant n'est pas au moins diacre:}

\versiculus{Dómine, exáudi oratiónem meam.}{Et clamor meus ad te véniat.}{Seigneur, exauce ma prière.}{Et que mon cri parvienne jusqu'à toi.}

\rubric{Puis il dit \normaltext{Orémus} et la collecte du jour et on répond \normaltext{Amen.}}

\versiculus{Dóminus vobíscum.}{Et cum spíritu tuo.}{Le Seigneur soit avec vous.}{Et avec votre esprit.}
\rubric{ou bien \normaltext{Dómine, exáudi}, etc.}\\

\rubric{Aux fêtes solennelles}
\gscore{ORBDa}{T}{}{Benedicamus Domino!Solennités}{\vvrub Bénissons le Seigneur. \rrrub Nous rendons grâces à Dieu.}

\vspace{\baselineskip}

\rubric{Aux autres fêtes de la Vierge}
\gscore{ORBDd}{T}{}{Benedicamus Domino!Fêtes de la Vierge}{\vvrub Bénissons le Seigneur. \rrrub Nous rendons grâces à Dieu.}

\pagebreak

\versiculus{Fidélium ánimæ per misericórdiam Dei requiéscant in pace.}{Amen.}{Que par la miséricorde de Dieu, les âmes des fidèles trépassés reposent en paix.}{Amen.}

\rubric{On reste en silence le temps d'un \normaltext{Pater}.}

\intermediatetitle{Après l'Office}

\begin{paracol}{2}
\lettrine{S}{acrosánctæ} et indivíduæ Trinitáti, crucifíxi Dómini nostri Jesu Christi humanitáti, beatíssimæ et gloriosíssimæ sempérque Vírginis Maríæ fœcúndæ integritáti, et ómnium Sanctórum universitáti sit sempitérna laus, honor, virtus et glória ab omni creatúra, nobísque remíssio ómnium peccatórum, per infiníta sǽcula sæculórum. Amen.

\switchcolumn
\vspace{-1mm}
\lettrine{À}{ la Très Sainte} et indivisible Trinité, 
à l'humanité de Notre Seigneur Jésus-Christ crucifié, 
et à la féconde intégrité de la Bienheureuse et très glorieuse Marie toujours Vierge, 
ainsi qu'à toute l'assemblée des Saints, soient éternelle louange, honneur, puissance et gloire 
de la part de toute créature, et à nous rémission de nos péchés, pour l'infinie durée des siècles. Amen.

\end{paracol}

\vfill

\feast{TCPA}{Tons des \emph{pneumata} à la fin des antiennes}
	{Tons communs}{Tons communs}{2}{}{}{}{}{}{}
\rubric{Les \emph{pneumata} peuvent être ajoutés à la fin de certaines antiennes, aux grandes fêtes, là où c'est la coutume.}
\gscore[n]{ORAL3}{T}{}{Tons des \emph{pneumata} à la fin des antiennes}{}

\vfill

\pagebreak

\end{document}